%%%%%%%%%%%%%%%%%%%%%%%%
%
%   Thesis template by Youssif Al-Nashif
%
%   May 2020
%
%%%%%%%%%%%%%%%%%%%%%%%%

\chapter{Introduction}
\label{introduction}


% Big Ideas.

\hspace*{0.5cm} Learning Analytics is a field existing at the cross-section of machine-learning and data analytics with education. 
Many different techniques exist to leverage data from coursework or user interactions from multiple different user-types at universities and enable predictive 
or inferential capabilities. However, the implementation of these techniques are often overlooked or not offered by universities, which can result in problems. 
For instance, students that may join classes intended for a wider set of majors/programs tend to have differing levels of programming experience that result in bottlenecks in teaching and topics covered. 

For this, a network analysis of skills acquired in student’s previous coursework was investigated with a focus on creating concept-courses association graphs using 
a variety of distance metrics.  These distance metrics include cosine similarity,  Jaccard's index, Burrow's Delta, Argamon's Linear Delta and others \cite{lan_tag-aware_2014} 
\cite{lan_sparse_nodate}  that were implemented on Florida Polytechnic University course data, focusing on the contents in each course and the overlap of topics. 

\indent This technique was then used with dimensionality reduction techniques such as Multiple Correspondence Analysis (MCA) and MDS to generate biplots where 
possible clusters of concepts may be identified.  Initial focus was on the Data Science and Business Analytics curriculum, but can be broadened to include other plans of 
study once the methods are further developed. The investigation involved the concepts and methods described on both course descriptions and course outlines using both a 
bigram approach along with an approach using the full text passages of each respectively.  The final goal was to create reusable and reproducible templates for easy successive 
iterations. 
