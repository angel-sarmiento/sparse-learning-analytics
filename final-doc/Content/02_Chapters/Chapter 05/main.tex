%%%%%%%%%%%%%%%%%%%%%%%%
%
%   Thesis template by Youssif Al-Nashif
%
%   May 2020
%
%%%%%%%%%%%%%%%%%%%%%%%%

\chapter{Conclusion}

Learning Analytics and Unsupervised Learning methods were used as basis for developing course-concept association graphs. 
This work displayed these associations after a preliminary text mining analysis via the creation of document-term matrices 
on course catalog data.  This then evolved into creating clusters using dimensionality reduction methods such as MCA to 
further define course-concept relations, as well as set the groundwork for more  micro-analyses in further iterations by 
creating distance matrices with a variety of distance metrics.  There were numerous interesting takeaways gained from the 
analyses performed here, as well as many different downsides to each approach. However we see the results here as an 
overall positive for future degree design.  Additional analysis is needed to provide sufficient evidence to suggest 
non-linear course design,  however the groundwork is there for future analyses to determine so.

We believe this  project resulted in contributions to the existing body of knowledge on these techniques applied to 
learning analytics, as well as to the University as a whole in providing a tangible body of work to be used by future 
students, faculty, and course planners.   The templates and code produced here can be easily applied to other 
degree programs like Computer Science and Mechanical Engineering by just changing a single line of code. The code 
included also has measures in place for maintaining dependencies reliably,  ensuring that the work is reproducible 
with minimal effort. With Learning Analytics being an emergent field, relevant research to undertake, helping to add 
to the body of knowledge and encouraging the use of modern data-driven curriculum design at Florida Polytechnic University. 

nocite{*}






